
\documentclass[11pt]{beamer}
\usepackage[utf8]{inputenc}
\usepackage[T1]{fontenc}
\usepackage{lmodern}
\usetheme{Copenhagen}
\usepackage{listings}
\usepackage{color}
\usepackage{caption}
\usepackage{graphicx}
\usepackage{outlines}
\setbeamertemplate{caption}[numbered]

\definecolor{dkgreen}{rgb}{0,0.6,0}
\definecolor{gray}{rgb}{0.5,0.5,0.5}
\definecolor{mauve}{rgb}{0.58,0,0.82}



% Code
\usepackage{courier} %% Sets font for listing as Courier.
\usepackage{listings, xcolor}
\lstset{
	tabsize = 4, %% set tab space width
	showstringspaces = false, %% prevent space marking in strings, string is defined as the text that is generally printed directly to the console
	numbers = left, %% display line numbers on the left
	commentstyle = \color{red}, %% set comment color
	keywordstyle = \color{blue}, %% set keyword color
	stringstyle = \color{red}, %% set string color
	rulecolor = \color{black}, %% set frame color to avoid being affected by text color
	basicstyle = \small \ttfamily , %% set listing font and size
	breaklines = true, %% enable line breaking
	numberstyle = \tiny,
}

%Page Number
\addtobeamertemplate{navigation symbols}{}{%
	\usebeamerfont{footline}%
	\usebeamercolor[fg]{footline}%
	\hspace{1em}%
	\insertframenumber/\inserttotalframenumber
}
\expandafter\def\expandafter\insertshorttitle\expandafter{%
	\insertshorttitle\hfill%
	\insertframenumber\,/\,\inserttotalframenumber}

\begin{document}
	\author{Ms. Sonam Wangmo}
	\title{ITS202: Algorithms and Data Structures}
	\subtitle{Advanced Data Structures}
	\institute{
		\textcolor{blue}{Gyalpozhing College of Information Technology \\ Royal University of Bhutan} \\
		\vspace{0.5cm}
	}
	%\date{}
	\setbeamercovered{transparent}
	%\setbeamertemplate{navigation symbols}{}
	\begin{frame}[plain]
		\maketitle 
	\end{frame}

	\begin{frame}
		\frametitle{Balanced Search Trees: AVL}
	    \begin{block}{Height-Balance Property or Balance Factor}
	    	For every internal position p of T , the heights of the children of p differ by at most 1.\\
	    	\alert{Balance}\\
	    	Balance factor= height(left subtree) - height(right subtree).
	    \end{block}
      \begin{block}{AVL}
      	Any binary search tree T that satisfies the height-balance property is said to be an \textcolor{blue}{AVL} tree, named after the initials of its inventors: \textbf{Adel’son-Vel’skii and Landis}.\\
      	\alert{|B(n)| <= 1}
      \end{block}
	\end{frame}	

    \begin{frame}
    	\frametitle{Balanced Search Trees: AVL}
        \begin{figure}
        	\centering
        	\includegraphics[width=0.8\linewidth]{"Screenshot 2020-11-19 at 3.10.01 PM"}
        	\caption{An example of an AVL tree}
        	\label{fig:screenshot-2020-11-19-at-3}
        \end{figure}
    \end{frame}	
 
  \begin{frame}
    \frametitle{Balanced Search Trees: AVL}
	       \begin{block}{Height of a node}
	       	Height of longest path from it down to a leaf.\\
	       	H($\emptyset$) = -1\\
	       	H(Single Node) = 0\\
	       	Height of a node = Maximum[height(leftchild), height(right child)]+1\\
	       	\end{block}
       	  \alert{Worsecase is when the right subtree has height 1 more than left subtree for every node.}
   \end{frame}	

      \begin{frame}
    	\frametitle{Balanced Search Trees: AVL}
    	\begin{figure}
    		\centering
    		\includegraphics[width=0.8\linewidth]{"Screenshot 2020-11-21 at 12.21.29 AM"}
    		\caption{Balanced BST:AVL}
    		\label{fig:screenshot-2020-11-21-at-12}
    	\end{figure} 
      \alert{Note:}
       \textcolor{blue}{Positive Balance: Left-Heavy}\\	
        \textcolor{blue}{Negative Balance: Right-Heavy}\\	
         \textcolor{blue}{B(n)=H($T_{L}$)  - H($T_{R}$) }
    \end{frame}	
    
    \begin{frame}
    	\frametitle{Insertion in AVL Tree}
    	\begin{enumerate}
    		\item Simple BST indert
    		\item Fix the AVL property from changed node up.
    		\item 
    	\end{enumerate}
        \alert{	Note: Inserting a new node can cause the balance factor of some node to become 2 or -2. In that case, we fix the balance factors by use of rotations.}
    
    \end{frame}	

    \begin{frame}
    	\frametitle{AVL Tree: Rotations}
    	\textcolor{blue}{Rotation means maintaining the BST invariant at the same time as maintaing the balance threshold.}\\
    	Rotations fix imbalance.\\
    	\begin{block}{Left and Right Rotations}
    		\alert{Left-Heavy}
    		\begin{enumerate}
    			\item Right rotation
    			\item Left-Right rotation
    		\end{enumerate}
    		\alert{Right-Heavy}
    		\begin{enumerate}
    			\item Left rotation
    			\item Right-Left rotation
    		\end{enumerate}
    	\end{block}
    \end{frame}	

     \begin{frame}
    	\frametitle{Rotations: Right}
    	\textcolor{blue}{When we do rotations, focus on 2 nodes, x and y}\\
    	\alert{Insert 3,2,1}
        \begin{figure}
        	\centering
        	\includegraphics[width=0.5\linewidth]{"Screenshot 2020-11-24 at 10.21.20 AM"}
        	\caption{BST}
        	\label{fig:screenshot-2020-11-24-at-10}
        \end{figure}
        
    \end{frame}	

     \begin{frame}
     	\frametitle{Rotations: Right}
     	\begin{figure}
     		\centering
     		\includegraphics[width=0.8\linewidth]{"Screenshot 2020-11-24 at 10.26.23 AM"}
     		\caption{AVL Tree}
     		\label{fig:screenshot-2020-11-24-at-10}
     	\end{figure}	
     \end{frame}
 
       \begin{frame}
    	\frametitle{Rotations: Left-Right}
    	\textcolor{blue}{When we do rotations, focus on 2 nodes, x and y}\\
    	\alert{Insert 3,1,2}
    	\begin{figure}
    		\centering
    		\includegraphics[width=0.9\linewidth]{"Screenshot 2020-11-24 at 11.12.17 AM"}
    		\caption{Right rotation}
    		\label{fig:screenshot-2020-11-24-at-11}
    	\end{figure}   	
    \end{frame}	

        \begin{frame}
 	\frametitle{Rotations: Left-Right}
    \begin{figure}
    	\centering
    	\includegraphics[width=1\linewidth]{"Screenshot 2020-11-24 at 11.20.53 AM"}
    	\caption{AVL Tree: Left-right rotation}
    	\label{fig:screenshot-2020-11-24-at-11}
    \end{figure}    
 \end{frame}

 \begin{frame}
	\frametitle{Rotations: Left}
	\textcolor{blue}{When we do rotations, focus on 2 nodes, x and y}\\
	\alert{Insert 1,2,3}
	\begin{figure}
		\centering
		\includegraphics[width=1\linewidth]{"Screenshot 2020-11-24 at 11.32.22 AM"}
		\caption{AVL: Left rotation}
		\label{fig:screenshot-2020-11-24-at-11}
	\end{figure}	
\end{frame}	

 \begin{frame}
	\frametitle{Rotations: Right-Left}
   \textcolor{blue}{When we do rotations, focus on 2 nodes, x and y}\\
   \alert{Insert 1,3,2}
  	\begin{figure}
  		\centering
  		\includegraphics[width=0.9\linewidth]{"Screenshot 2020-11-24 at 11.40.09 AM"}
  		\caption{Left rotation:AVL}
  		\label{fig:screenshot-2020-11-24-at-11}
  	\end{figure} 	
\end{frame}	

 \begin{frame}
	\frametitle{Rotations: Right-Left}
    \begin{figure}
    	\centering
    	\includegraphics[width=1\linewidth]{"Screenshot 2020-11-24 at 11.47.36 AM"}
    	\caption{AVL Tree: Right-left Rotation}
    	\label{fig:screenshot-2020-11-24-at-11}
    \end{figure}   	
\end{frame}	


\begin{frame}
	\frametitle{Balanced Search Trees:  2-3 search trees}
	\textcolor{blue}{ Allow 1 or 2 keys per node.}\\
	\begin{itemize}
		\item 2-node: one key, two children. 
		\item 3-node: two keys, three children.
	\end{itemize}
	\begin{figure}
		\centering
		\includegraphics[width=0.8\linewidth]{"Screenshot 2020-11-24 at 5.50.40 PM"}
		\caption{2-3 search representation}
		\label{fig:screenshot-2020-11-24-at-5}
	\end{figure}
\end{frame}		

\begin{frame}
	\frametitle{Balanced Search Trees:  red-black BSTs }
	\begin{figure}
		\centering
		\includegraphics[width=0.6\linewidth]{"Screenshot 2020-11-24 at 6.20.09 PM"}
		\caption{Red-Black Tree}
		\label{fig:screenshot-2020-11-24-at-6}
	\end{figure}
	
\end{frame}	
\begin{frame}
	\frametitle{Balanced Search Trees:  B-trees(Bayer-McCreight, 1972)}
	\begin{figure}
		\centering
		\includegraphics[width=1.05\linewidth]{"Screenshot 2020-11-24 at 6.22.16 PM"}
		\caption{B tree}
		\label{fig:screenshot-2020-11-24-at-6}
	\end{figure}
\end{frame}	

\begin{frame}
	\frametitle{Balanced Search Trees:  Performance}
	\alert{Bottom line. Guaranteed logarithmic performance for search and insert. O(Log n)}
\end{frame}		
\end{document}