
\documentclass[11pt]{beamer}
\usepackage[utf8]{inputenc}
\usepackage[T1]{fontenc}
\usepackage{lmodern}
\usetheme{Copenhagen}
\usepackage{listings}
\usepackage{color}
\usepackage{caption}
\usepackage{graphicx}
\usepackage{outlines}
\setbeamertemplate{caption}[numbered]

\definecolor{dkgreen}{rgb}{0,0.6,0}
\definecolor{gray}{rgb}{0.5,0.5,0.5}
\definecolor{mauve}{rgb}{0.58,0,0.82}



% Code
\usepackage{courier} %% Sets font for listing as Courier.
\usepackage{listings, xcolor}
\lstset{
	tabsize = 4, %% set tab space width
	showstringspaces = false, %% prevent space marking in strings, string is defined as the text that is generally printed directly to the console
	numbers = left, %% display line numbers on the left
	commentstyle = \color{red}, %% set comment color
	keywordstyle = \color{blue}, %% set keyword color
	stringstyle = \color{red}, %% set string color
	rulecolor = \color{black}, %% set frame color to avoid being affected by text color
	basicstyle = \small \ttfamily , %% set listing font and size
	breaklines = true, %% enable line breaking
	numberstyle = \tiny,
}

%Page Number
\addtobeamertemplate{navigation symbols}{}{%
	\usebeamerfont{footline}%
	\usebeamercolor[fg]{footline}%
	\hspace{1em}%
	\insertframenumber/\inserttotalframenumber
}
\expandafter\def\expandafter\insertshorttitle\expandafter{%
	\insertshorttitle\hfill%
	\insertframenumber\,/\,\inserttotalframenumber}

\begin{document}
	\author{Ms. Sonam Wangmo}
	\title{ITS202: Algorithms and Data Structures}
	\subtitle{Advanced Data Structures}
	\institute{
		\textcolor{blue}{Gyalpozhing College of Information Technology \\ Royal University of Bhutan} \\
		\vspace{0.5cm}
	}
	%\date{}
	\setbeamercovered{transparent}
	%\setbeamertemplate{navigation symbols}{}
	\begin{frame}[plain]
		\maketitle
	\end{frame}

	\begin{frame}
		\frametitle{BST Ordered Operations: Keys()}
	      \textbf{BST Traversal: }\textcolor{blue}{Inorder Traversal}
	      \begin{enumerate}
	      	\item Print all the keys in the left subtree (which are less than the key at the root by definition of BSTs)
	      	\item Then print the key at the root,
	      	\item Then print all the keys in the right subtree (which are greater than the key at the root by definition of BSTs)
	      \end{enumerate}
	\end{frame}	

    \begin{frame}
    	\frametitle{BST Ordered Operations: Keys()}
    	\textbf{BST Traversal: }\textcolor{blue}{Inorder Traversal}
    	\begin{enumerate}
    		\item Traverse Left Subtree
    		\item Enqueue Key 
    		\item Traverse Right Subtree
    	\end{enumerate}
    \end{frame}	
    
     \begin{frame}
    	\frametitle{BST Ordered Operations: Keys()}
         \begin{figure}
         	\centering
         	\includegraphics[width=0.9\linewidth]{"Screenshot 2020-11-13 at 4.09.37 PM"}
         	\caption{keys() method}
         	\label{fig:screenshot-2020-11-13-at-4}
         \end{figure}
        \textcolor{blue}{Property: } Inorder traversal of a BST yields keys in ascending order.
    \end{frame}	

     \begin{frame}
    	\frametitle{\textcolor{red}{Analysis:} How efficient are the order-based operations in BSTs ?} 
    	\alert{ In a BST,all operations take time proportional to the height of the tree, in the worst case.}
    \end{frame}	

    

      \begin{frame}
    	\frametitle{BST: ordered symbol table operations summary} 
    	\begin{figure}
    		\centering
    		\includegraphics[width=1.1\linewidth]{"Screenshot 2020-11-13 at 8.43.37 PM"}
    		\caption{Summary}
    		\label{fig:screenshot-2020-11-13-at-8}
    	\end{figure}  	
    \end{frame}	

    \begin{frame}
	   	\frametitle{BST: Deletion} 
	   	 \textcolor{blue}{ Predecessor and Successor Concepts}\\
	   	 Where is the predecessor of a node in a tree􏰎, assuming all keys are distinct􏰀?
	    \begin{figure}
	    	\centering
	    	\includegraphics[width=0.7\linewidth]{"Screenshot 2020-11-15 at 9.48.41 PM"}
	    	\caption{BST}
	    	\label{}
	    \end{figure}
   \end{frame}	

     \begin{frame}
    	\frametitle{BST: Deletion} 
    	\textcolor{blue}{ Predecessor and Successor Concepts}\\
    	If X has two children􏰎 its predecessor is value in its left subtree and its successor value in its right subtree􏰈. 
    	\begin{figure}
    		\centering
    		\includegraphics[width=0.3\linewidth]{"Screenshot 2020-11-15 at 9.54.39 PM"}
    		\caption{}
    		\label{fig:screenshot-2020-11-15-at-9}
    	\end{figure}
    	
    	\alert {If it does not have a left child􏰎, a node􏰐s predecessor is its fi􏰂rst left ancestor}
    \end{frame}	

   \begin{frame}
   	\frametitle{BST: Deletion} 
   	 \begin{block}{Delete the minimum/maximum}
   	 deleteMin(): Remove the key-value pair with the smallest key.
   	 \end{block}
     \textcolor{blue}{To delete the minimum key:}
    \begin{enumerate}
    	\item Go left until finding a node with a null left link
    	\item Replace that node by its right link.
    	\item Update subtree counts.
    \end{enumerate}	   
   \end{frame}	

    \begin{frame}
   	\frametitle{BST: Deletion} 
   	\begin{figure}
   		\centering
   		\includegraphics[width=0.38\linewidth]{"Screenshot 2020-11-16 at 9.41.40 PM"}
   		\caption{Deleting the minimum in a BST}
   		\label{fig:screenshot-2020-11-16-at-9}
   	\end{figure}  	
   \end{frame}	

   \begin{frame}
  	\frametitle{BST: Deletion} 
  	  \begin{figure}
  	  	\centering
  	  	\includegraphics[width=0.9\linewidth]{"Screenshot 2020-11-16 at 9.41.58 PM"}
  	  	\caption{deleteMin() method}
  	  	\label{fig:screenshot-2020-11-16-at-9}
  	  \end{figure}  
  \end{frame}	

   \begin{frame}
   	\frametitle{BST: Hibbard Deletion} 
   	To delete a node with key k: search for node t containing key k. \\
   	\textcolor{blue}{Case 0.} [0 children] Delete t by setting parent link to null.
   	\begin{figure}
   		\centering
   		\includegraphics[width=1\linewidth]{"Screenshot 2020-11-16 at 10.02.29 PM"}
   		\caption{Deletion in a BST}
   		\label{fig:screenshot-2020-11-16-at-10}
   	\end{figure}
   \end{frame}	

    \begin{frame}
   	\frametitle{BST: Hibbard Deletion} 
   	To delete a node with key k: search for node t containing key k.  \\
   	\textcolor{blue}{Case 1. [1 child]} Delete t by replacing parent link.
   \begin{figure}
   	\centering
   	\includegraphics[width=1.1\linewidth]{"Screenshot 2020-11-16 at 10.17.41 PM"}
   	\caption{Deletion in BST}
   	\label{fig:screenshot-2020-11-16-at-10}
   \end{figure}
   \end{frame}	

   \begin{frame}
  	\frametitle{BST: Hibbard Deletion} 
  	To delete a node with key k: search for node t containing key k. \\
  	\textcolor{blue}{Case 2. [2 children]} \\
  	\begin{enumerate}
  		\item Find successor x of t.  \alert{ <--- x has no left child}
  		\item Delete the minimum in t's right subtree.  \alert{ <---  but don't garbage collect x}
  		\item Put x in t's spot.  \alert{ <---  still a BST}
  	\end{enumerate}
  \end{frame}	

   \begin{frame}
  	\frametitle{BST: Hibbard Deletion} 
  	\begin{figure}
  		\centering
  		\includegraphics[width=0.9\linewidth]{"Screenshot 2020-11-16 at 11.45.48 PM"}
  		\caption{Deletion in BST}
  		\label{fig:screenshot-2020-11-16-at-11}
  	\end{figure}
  	
  \end{frame}	

  \begin{frame}
  	\frametitle{Hibbard deletion: Java implementation} 
  	\begin{figure}
  		\centering
  		\includegraphics[width=1\linewidth]{"Screenshot 2020-11-16 at 10.33.33 PM"}
  		\caption{}
  		\label{fig:screenshot-2020-11-16-at-10}
  	\end{figure}	
  \end{frame}	

 \begin{frame}
 	\frametitle{BST Time Complexity} 
 	\begin{figure}
 			\centering
 			\includegraphics[width=0.7\linewidth]{"Screenshot 2020-11-16 at 10.36.22 PM"}
 			\caption{}
 			\label{fig:screenshot-2020-11-16-at-10}
 	\end{figure}	
 \end{frame}	

 \begin{frame}
 	\alert{WHAT IS THE BAD NEWS???}	
 \end{frame}	
  
\end{document}