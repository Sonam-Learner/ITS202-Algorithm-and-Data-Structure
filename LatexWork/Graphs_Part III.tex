\documentclass[11pt]{beamer}
\usepackage[utf8]{inputenc}
\usepackage[T1]{fontenc}
\usepackage{lmodern}
\usetheme{Copenhagen}
\usepackage{listings}
\usepackage{color}
\usepackage{caption}
\usepackage{graphicx}
\usepackage{outlines}
\setbeamertemplate{caption}[numbered]

\definecolor{dkgreen}{rgb}{0,0.6,0}
\definecolor{gray}{rgb}{0.5,0.5,0.5}
\definecolor{mauve}{rgb}{0.58,0,0.82}


% Code
\usepackage{courier} %% Sets font for listing as Courier.
\usepackage{listings, xcolor}
\lstset{
	tabsize = 4, %% set tab space width
	showstringspaces = false, %% prevent space marking in strings, string is defined as the text that is generally printed directly to the console
	numbers = left, %% display line numbers on the left
	commentstyle = \color{red}, %% set comment color
	keywordstyle = \color{blue}, %% set keyword color
	stringstyle = \color{red}, %% set string color
	rulecolor = \color{black}, %% set frame color to avoid being affected by text color
	basicstyle = \small \ttfamily , %% set listing font and size
	breaklines = true, %% enable line breaking
	numberstyle = \tiny,
}

%Page Number
\addtobeamertemplate{navigation symbols}{}{%
	\usebeamerfont{footline}%
	\usebeamercolor[fg]{footline}%
	\hspace{1em}%
	\insertframenumber/\inserttotalframenumber
}
\expandafter\def\expandafter\insertshorttitle\expandafter{%
	\insertshorttitle\hfill%
	\insertframenumber\,/\,\inserttotalframenumber}
\begin{document}
	\author{Ms. Sonam Wangmo}
	\title{ITS202: Algorithms and Data Structure}
	\subtitle{Shortest Paths}
	\institute{
		\textcolor{blue}{Gyalpozhing College of Information Technology \\ Royal University of Bhutan} \\
		\vspace{0.5cm}
	}
	%\date{}
	\setbeamercovered{transparent}
	%\setbeamertemplate{navigation symbols}{}
	\begin{frame}[plain]
		\maketitle
	\end{frame}

  \begin{frame}
 	\frametitle	{Edge-Weighted Graphs} 
 	\alert{An edge-weighted graph is a graph model where we associate weights or costs with each edge.}\\
 	Types of Edge Weighted Graph
 	\begin{enumerate}
 		\item Edge Weighted Graph
 		\item Edge Weighted Digraph
 	\end{enumerate}  
 \end{frame}

  \begin{frame}
	\frametitle	{Weighted Undirected Edge API } 

	\begin{figure}
		\centering
		\includegraphics[width=1\linewidth]{"Screenshot 2020-12-06 at 3.14.21 PM"}
		\label{fig:screenshot-2020-12-06-at-3}
	\end{figure}
\end{frame}

\begin{frame}
	\frametitle	{Edge-Weighted graph API} 
	\begin{figure}
		\centering
		\includegraphics[width=1.05\linewidth]{"Screenshot 2020-12-06 at 3.22.56 PM"}
		\label{fig:screenshot-2020-12-06-at-3}
	\end{figure}	
\end{frame}

  \begin{frame}
	\frametitle	{Weighted Directed Edge API } 
	\begin{figure}
		\centering
		\includegraphics[width=1\linewidth]{"Screenshot 2020-12-06 at 3.40.42 PM"}
		\label{fig:screenshot-2020-12-06-at-3}
	\end{figure}    
\end{frame}

\begin{frame}
	\frametitle	{Edge-Weighted Digraph API} 
	\begin{figure}
		\centering
		\includegraphics[width=1.05\linewidth]{"Screenshot 2020-12-06 at 3.42.49 PM"}
		\label{fig:screenshot-2020-12-06-at-3}
	\end{figure}	
\end{frame}

\begin{frame}
	\frametitle	{Data structures for single-source shortest paths
	} 
   \textcolor{blue}{Goal.} Find the shortest path from s to every other vertex.
  \\
   Can represent the \textcolor{red}{shortest-paths tree} (SPT) with two vertex-indexed arrays: 
   \begin{itemize}
   	\item distTo[v] is length of shortest path from s to v.
   	\item edgeTo[v] is last edge on shortest path from s to v.
   \end{itemize}	
   \begin{figure}
   	\centering
   	\includegraphics[width=0.8\linewidth]{"Screenshot 2020-12-06 at 7.25.46 PM"}
   	\label{fig:screenshot-2020-12-06-at-7}
   \end{figure}  
\end{frame}

\begin{frame}
	\frametitle	{Edge Relaxation} 
	Relax \textcolor{blue}{edge e = v→w.}
	\begin{itemize}	
		\item distTo[v] is length of shortest \alert{known} path from s to v. 
		\item distTo[w] is length of shortest \alert{known} path from s to w. 
		\item edgeTo[w] is last edge on shortest \alert{known} path from s to w. 
		\item If \textcolor{blue}{e = v→w} gives shorter path to w through v,
		update both distTo[w] and edgeTo[w].
	\end{itemize}
  
\end{frame}

\begin{frame}
	\frametitle	{Edge Relaxation} 
	\textcolor{blue}{Relax(u,v,w)}
	\begin{itemize}	
		\item if d[v] >= d[u] + w(u,v)\\
		\item	--->	d[v] = d[u] + w(u,v)
	\end{itemize}
	\begin{figure}
		\centering
		\includegraphics[width=0.85\linewidth]{"Screenshot 2020-12-06 at 7.33.48 PM"}
		\label{fig:screenshot-2020-12-06-at-7}
	\end{figure}   
\end{frame}

\begin{frame}
	\frametitle	{Generic shortest-paths algorithm} 
    \alert{Generic algorithm (to compute SPT from s)}\\
    Initialize distTo[s] = 0 and distTo[v] = $ \infty$ for all other vertices.\\
    Repeat until optimality conditions are satisfied:\\
    ------>Relax any edge.\\
    \alert{Efficient implementations.} How to choose which edge to relax? 
    \\ 1. Dijkstra's algorithm (nonnegative weights).
    \\ 2. Bellman-Ford algorithm (nonnegative cycles and nonnegative weights).        
\end{frame}

\begin{frame}
	\frametitle	{Dijkstra's algorithm} 
	\begin{itemize}
		\item Consider vertices in increasing order of distance from s (non-tree vertex with the lowest distTo[] value).
	   \item Add vertex to tree and relax all edges pointing from that vertex.
	\end{itemize}    
\end{frame}

\begin{frame}
	\frametitle	{Dijkstra's algorithm} 
    \begin{figure}
    	\centering
    	\includegraphics[width=1.05\linewidth]{"Screenshot 2020-12-06 at 8.03.33 PM"}
    	\label{fig:screenshot-2020-12-06-at-8}
    \end{figure}   
\end{frame}

\begin{frame}
	\frametitle	{Dijkstra's algorithm} 
   \begin{figure}
   	\centering
   	\includegraphics[width=1.05\linewidth]{"Screenshot 2020-12-06 at 8.05.57 PM"}
   	\label{fig:screenshot-2020-12-06-at-8}
   \end{figure}   
\end{frame}

\begin{frame}
	\frametitle	{Shortest paths with negative weights: failed attempts}  
    \textcolor{blue}{Dijkstra.} Doesn’t work with negative edge weights.
    \begin{figure}
    	\centering
    	\includegraphics[width=1\linewidth]{"Screenshot 2020-12-06 at 8.24.22 PM"}
    	\label{fig:screenshot-2020-12-06-at-8}
    \end{figure}
    \textcolor{blue}{Conclusion.} Need a different algorithm.
\end{frame}

\begin{frame}
	\frametitle	{Negative cycles}  
   \textcolor{blue}{Def.} A \alert{negative cycle} is a directed cycle whose sum of edge weights is negative.
   \begin{figure}
   	\centering
   	\includegraphics[width=0.8\linewidth]{"Screenshot 2020-12-06 at 8.44.01 PM"}
   	\label{fig:screenshot-2020-12-06-at-8}
   \end{figure}  
\end{frame}

\begin{frame}
	\frametitle	{Bellman-Ford algorithm}  
	\begin{figure}
		\centering
		\includegraphics[width=1\linewidth]{"Screenshot 2020-12-06 at 8.52.34 PM"}
		\label{fig:screenshot-2020-12-06-at-8}
	\end{figure}
\end{frame}

\begin{frame}
	\frametitle	{Bellman-Ford algorithm demo}  
    \begin{figure}
    	\centering
    	\includegraphics[width=1\linewidth]{"Screenshot 2020-12-06 at 8.54.17 PM"}
    	\label{fig:screenshot-2020-12-06-at-8}
    \end{figure}   
\end{frame}

\begin{frame}
	\frametitle	{Bellman-Ford algorithm demo}  
    \begin{figure}
    	\centering
    	\includegraphics[width=1.05\linewidth]{"Screenshot 2020-12-06 at 8.56.04 PM"}
    	\label{fig:screenshot-2020-12-06-at-8}
    \end{figure}    
\end{frame}
\end{document}