\documentclass[11pt]{beamer}
\usepackage[utf8]{inputenc}
\usepackage[T1]{fontenc}
\usepackage{lmodern}
\usetheme{Copenhagen}
\usepackage{listings}
\usepackage{color}
\usepackage{caption}
\usepackage{graphicx}
\usepackage{outlines}
\setbeamertemplate{caption}[numbered]

\definecolor{dkgreen}{rgb}{0,0.6,0}
\definecolor{gray}{rgb}{0.5,0.5,0.5}
\definecolor{mauve}{rgb}{0.58,0,0.82}


% Code
\usepackage{courier} %% Sets font for listing as Courier.
\usepackage{listings, xcolor}
\lstset{
	tabsize = 4, %% set tab space width
	showstringspaces = false, %% prevent space marking in strings, string is defined as the text that is generally printed directly to the console
	numbers = left, %% display line numbers on the left
	commentstyle = \color{red}, %% set comment color
	keywordstyle = \color{blue}, %% set keyword color
	stringstyle = \color{red}, %% set string color
	rulecolor = \color{black}, %% set frame color to avoid being affected by text color
	basicstyle = \small \ttfamily , %% set listing font and size
	breaklines = true, %% enable line breaking
	numberstyle = \tiny,
}


\begin{document}
	\author{Ms. Sonam Wangmo}
	\title{ITS202: Algorithms and Data Structures}
	\subtitle{Advanced Data Structures}
	\institute{
		\textcolor{blue}{Gyalpozhing College of Information Technology \\ Royal University of Bhutan} \\
		\vspace{0.5cm}
	}
	%\date{}
	\setbeamercovered{transparent}
	%\setbeamertemplate{navigation symbols}{}
	\begin{frame}[plain]
		\maketitle
	\end{frame}
	\begin{frame}
		\frametitle{Search Trees}
		\begin{enumerate}
			\item Binary Search trees
			\item Balanaced Binary Trees
			\begin{enumerate}
				\item Red Black Trees
				\item AVL trees
				\item B trees
			\end{enumerate}
		\end{enumerate}
	\end{frame}
    \begin{frame}
    	\frametitle{Binary Search Trees}
    		\begin{enumerate}
    		\item Binary Search Trees
    		\item Ordered Operations
    	    \item Deletion
    		\end{enumerate}
    \end{frame}
   
    \begin{frame}
   	\frametitle{Binary Search Trees}
   	\begin{block}{Definition}
   		\textbf{Binary Search trees} is a binary tree T with each position p storing a key-value pair(k,v) such that :
   		\begin{itemize}
   			\item Keys stored in the left subtree of p are less than k.
   			\item Keys stored in the right subtree of p are greater than k.
   		\end{itemize}
   	\end{block}
   \end{frame}

     \begin{frame}
    	\frametitle{Binary Search Trees}
     \begin{figure}
    	\centering
    	\includegraphics[width=0.9\linewidth]{"Screenshot 2020-11-02 at 10.08.12 AM"}
    	\caption{Binary Tree example}
    	\label{fig:screenshot-2020-11-02-at-10}
    \end{figure}
    \end{frame}
  
   
     \begin{frame}
    	\frametitle{Binary Search Trees}
    	\begin{figure}
    		\centering
    		\includegraphics[width=0.5\linewidth]{"Screenshot 2020-11-02 at 10.01.34 AM"}
    		\caption{Binary Search Tree}
    		\label{fig:screenshot-2020-11-02-at-10}
    	\end{figure}
    	\begin{alertblock}{Note}
    		As a matter of convenience, we will not diagram the values associated with keys, since those values do not affect the placement of items within a search tree.
    	\end{alertblock}
    \end{frame}
    
     \begin{frame}
    	\frametitle{BST representation in Java}
        \begin{block}{A Node is composed of four fields:}
        	\begin{itemize}
        		\item A Key and a Value.
        		\item A reference to the left and right subtree.
        	\end{itemize}
        \end{block}
       \begin{figure}
       	\centering
       	\includegraphics[width=0.8\linewidth]{"Screenshot 2020-11-02 at 10.21.59 AM"}
       	\caption{BST representation}
       	\label{fig:screenshot-2020-11-02-at-10}
       \end{figure}   
    \end{frame}
     
      \begin{frame}
     	\frametitle{BST representation in Java}
     	\begin{block}{Node count N}
     	The instance variable N gives the node count in the subtree rooted at the node.\\
     	\textcolor{blue}{size(x) = size(x.left) + size(x.right) + 1}
     	\end{block}
        \begin{figure}
        	\centering
        	\includegraphics[width=0.5\linewidth]{"Screenshot 2020-11-02 at 9.56.48 PM"}
        	\caption{BST}
        	\label{fig:screenshot-2020-11-02-at-9}
        \end{figure}
        
     \end{frame}
   
      \begin{frame}[fragile]
     	\frametitle{BST representation in Java}
     	 \begin{block}{Node class in BST}
    	\begin{lstlisting}[language=Java]
     	 private class Node
     	 	{
     	 			private Key key;
     	 			private Value val;
     	 			private Node left, right;
     	 			public Node(Key key, Value val)
     	 			{
     	 				this.key = key;
     	 				this.val = val;
     	 			}
     	 	}
     	\end{lstlisting}
     	 \end{block}
         \alert{Key and Value are generic types; Key is Comparable}
     \end{frame}
 
      \begin{frame}[fragile]
     	\frametitle{BST Implementation(Skeleton)}
      	\begin{lstlisting}[language=Java]
        public class BST<Key extends Comparable<Key>, Value>
        {
        	private Node root;
        	private class Node
           {  /* see previous slide */  }
        	public void put(Key key, Value val)
         	{  /* see next slides */  }
        	public Value get(Key key)
        	{  /* see next slides */  }
        	public void delete(Key key)
        	{  /* see next slides */  }
        	public Iterable<Key> iterator()
        	{  /* see next slides */  }
      	}
   	\end{lstlisting}
     \end{frame}
 
    \begin{frame}[fragile]
    	\frametitle{Binary Search Tree Demo}
    	\begin{block}{Note}
    		 If a node containing the key is in the table, we have a search hit, so we return the associated value. Otherwise, we have a search miss (and return null).\\
    		 	\textcolor{blue}{Search: }If less, go left; if greater, go right; if equal, search hit.
    	\end{block}
        \begin{figure}
        	\centering
        	\includegraphics[width=0.6\linewidth]{"Screenshot 2020-11-02 at 10.56.03 AM"}
        	\caption{BST Search demo}
        	\label{fig:screenshot-2020-11-02-at-10}
        \end{figure}    
    \end{frame}

     \begin{frame}[fragile]
    	\frametitle{Binary Search Tree Demo}
        \begin{figure}
        	\centering
        	\includegraphics[width=0.9\linewidth]{"Screenshot 2020-11-02 at 12.20.42 PM"}
        	\caption{Search hit (left) and search miss (right) in a BST}
        	\label{fig:screenshot-2020-11-02-at-12}
        \end{figure}
          
    \end{frame}
    
    \begin{frame}[fragile]
    	\frametitle{Binary Search Tree Demo}
    	\textcolor{blue}{Insert: }If less, go left; if greater, go right; if null, insert.
    	\begin{figure}
    		\centering
    		\includegraphics[width=0.9\linewidth]{"Screenshot 2020-11-02 at 10.58.51 AM"}
    		\caption{BST Insert demo}
    		\label{fig:screenshot-2020-11-02-at-10}
    	\end{figure}	
    \end{frame}

       \begin{frame}[fragile]
    	\frametitle{BST Search: Java Implementation}
    	\textcolor{blue}{Get: } Return value corresponding to given key, or null if no such key.
    	\begin{lstlisting}[language=Java]
        public Value get(Key key)
         {
         	Node x = root;
         	while (x != null)
         	{
         		int cmp = key.compareTo(x.key);
         		if      (cmp  < 0) x = x.left;
         		else if (cmp  > 0) x = x.right;
         		else return x.val;
         	}
         	return null;
         }
    	\end{lstlisting}
        \alert{Cost: Number of compares is equal to 1 + depth of node.}
    \end{frame}

     \begin{frame}[fragile]
     	\frametitle{BST Insert}
     	\textbf{\textcolor{blue}{Put:} }Associate value with key.\\
     	\begin{block}{Search for key, then two cases: } 
         	\\ Key in tree : reset value.
     		\\ Key not in tree : add new node.
     	\end{block}
     \end{frame}
 
      \begin{frame}[fragile]
     	\frametitle{BST Insert}
     	 \begin{figure}
     		\centering
     		\includegraphics[width=0.4\linewidth]{"Screenshot 2020-11-02 at 10.34.03 PM"}
     		\caption{Insertion into a BST}
     		\label{fig:screenshot-2020-11-02-at-10}
     	\end{figure}
     \end{frame}
 
    \begin{frame}[fragile]
    	\frametitle{BST Insert}
        \begin{figure}
        	\centering
        	\includegraphics[width=0.55\linewidth]{"Screenshot 2020-11-02 at 10.39.30 PM"}
        	\caption{BST trace for standard indexing client}
        	\label{fig:screenshot-2020-11-02-at-10}
        \end{figure}
        
    \end{frame}

      \begin{frame}[fragile]
    	\frametitle{BST Insert: Implementation}
    	\begin{figure}
    		\centering
    		\includegraphics[width=0.9\linewidth]{"Screenshot 2020-11-02 at 10.45.02 PM"}
    		\caption{}
    		\label{fig:screenshot-2020-11-02-at-10}
    	\end{figure}	
    \end{frame}

     \begin{frame}[fragile]
    	\frametitle{BST Running Time Analysis}
    	\begin{block}{Search and Insert Operation in BST}
    		\textcolor{blue}{Best Case:} $\omega$(log n) or  $\omega$(1)\\
    		\textcolor{blue}{Worst Case:} O(n)
    	\end{block}
        \begin{figure}
        	\centering
        	\includegraphics[width=1.001\linewidth]{"Screenshot 2020-11-03 at 10.19.11 PM"}
        	\caption{Tree Shape}
        	\label{fig:screenshot-2020-11-03-at-10}
        \end{figure}
        \alert{The running times of algorithms on binary
        	search trees depend on the shapes of the trees, which, in turn, depend on the order in which keys are inserted.}
    \end{frame}

    \begin{frame}[fragile]
    	\frametitle{BST Ordered Operations}
    	\begin{exampleblock}{Minimum}
    		If the left link of the root is null, the smallest key in a BST is the key at the root; if the left link is not null, the smallest key in the BST is the smallest key in the subtree rooted at the node referenced by the left link.
    	\end{exampleblock}
        \begin{exampleblock}{Maximum}
        	Finding the maximum key is similar, moving to the right instead of to the left.
        \end{exampleblock}
    \end{frame}
 
    \begin{frame}[fragile]
  	\frametitle{BST Ordered Operations}
  	\begin{figure}
  		\centering
  		\includegraphics[width=0.8\linewidth]{"Screenshot 2020-11-03 at 10.59.38 PM"}
  		\caption{}
  		\label{fig:screenshot-2020-11-03-at-10}
  	\end{figure}	
  \end{frame}
      \begin{frame}[fragile]
     	\frametitle{BST Ordered Operations}
        \begin{figure}
        	\centering
        	\includegraphics[width=0.9\linewidth]{"Screenshot 2020-11-03 at 10.39.52 PM"}
        	\caption{Code for Min()}
        	\label{fig:screenshot-2020-11-03-at-10}
        \end{figure}
        
     \end{frame}
      
   \begin{frame}[fragile]
   	\frametitle{BST Ordered Operations:  Floor and ceiling}
     \begin{exampleblock}{ Floor and ceiling}
     	Floor: Largest key $<= $ a given key. \\
     	Ceiling: Smallest key $>=$ a given key.
     \end{exampleblock}
   	\begin{figure}
   		\centering
   		\includegraphics[width=0.5\linewidth]{"Screenshot 2020-11-03 at 11.01.57 PM"}
   		\caption{}
   		\label{fig:screenshot-2020-11-03-at-11}
   	\end{figure}
   	\alert{Q. How to find the floor / ceiling?}
   \end{frame}
   
    \begin{frame}[fragile]
	   	\frametitle{BST Ordered Operations:  Computing the floor}
	   \begin{figure}
	   	\centering
	   	\includegraphics[width=0.9\linewidth]{"Screenshot 2020-11-03 at 11.15.58 PM"}
	   	\label{fig:screenshot-2020-11-03-at-11}
	   \end{figure}
   \end{frame}

    \begin{frame}[fragile]
   	\frametitle{BST Ordered Operations:  Computing the floor}
     \begin{figure}
     	\centering
     	\includegraphics[width=0.34\linewidth]{"Screenshot 2020-11-03 at 11.18.17 PM"}
     	\caption{Computing the floor function}
     	\label{fig:screenshot-2020-11-03-at-11}
     \end{figure}
     
   \end{frame}

   \begin{frame}[fragile]
   	\frametitle{BST Ordered Operations:  Computing the floor}
   	\begin{figure}
   		\centering
   		\includegraphics[width=0.8\linewidth]{"Screenshot 2020-11-03 at 11.23.07 PM"}
   		\caption{Code for floor function}
   		\label{fig:screenshot-2020-11-03-at-11}
   	\end{figure}	
   \end{frame}

    \begin{frame}[fragile]
   	\frametitle{BST Ordered Operations:  Rank and Select}
    \textcolor{blue}{Q. }How to implement rank() and select() efficiently?\\
    \textcolor{blue}{A. } In each node, we store the number of nodes in the subtree rooted at that node; to implement size(), return the count at the root. 
     \begin{figure}
     	\centering
     	\includegraphics[width=0.6\linewidth]{"Screenshot 2020-11-04 at 9.27.23 AM"}
     	\caption{BST}
     	\label{fig:screenshot-2020-11-04-at-9}
     \end{figure}
   \end{frame}

     \begin{frame}[fragile]
    	\frametitle{BST Ordered implementation: subtree counts}
    	\begin{figure}
    		\centering
    		\includegraphics[width=0.5\linewidth]{"Screenshot 2020-11-04 at 9.38.24 AM"}
    		\caption{Node class}
    		\label{fig:screenshot-2020-11-04-at-9}
    	\end{figure}
    	\begin{figure}
    		\centering
    		\includegraphics[width=0.5\linewidth]{"Screenshot 2020-11-04 at 9.40.05 AM"}
    		\caption{Size Method}
    		\label{fig:screenshot-2020-11-04-at-9}
    	\end{figure}	
    \end{frame}

    \begin{frame}[fragile]
    	\frametitle{BST Ordered implementation: subtree counts}
    	\begin{figure}
    		\centering
    		\includegraphics[width=1\linewidth]{"Screenshot 2020-11-04 at 10.44.39 AM"}
    		\caption{Put Method}
    		\label{fig:screenshot-2020-11-04-at-10}
    	\end{figure}
    \end{frame}

    \begin{frame}[fragile]
    	\frametitle{BST Ordered implementation: Rank}
       \textcolor{blue}{Rank. }	How many keys  $< $ k ?
       \begin{figure}
       	\centering
       	\includegraphics[width=0.7\linewidth]{"Screenshot 2020-11-04 at 10.54.45 AM"}
       	\caption{BST}
       	\label{fig:screenshot-2020-11-04-at-10}
       \end{figure}
    \end{frame}

    \begin{frame}[fragile]
    	\frametitle{BST Ordered implementation: Rank}
    	\begin{figure}
    		\centering
    		\includegraphics[width=1.05\linewidth]{"Screenshot 2020-11-04 at 10.56.40 AM"}
    		\caption{rank method}
    		\label{fig:screenshot-2020-11-04-at-10}
    	\end{figure}
    	
    \end{frame}
  
\end{document}