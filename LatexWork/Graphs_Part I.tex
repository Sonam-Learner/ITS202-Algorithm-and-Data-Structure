\documentclass[11pt]{beamer}
\usepackage[utf8]{inputenc}
\usepackage[T1]{fontenc}
\usepackage{lmodern}
\usetheme{Copenhagen}
\usepackage{listings}
\usepackage{color}
\usepackage{caption}
\usepackage{graphicx}
\usepackage{outlines}
\setbeamertemplate{caption}[numbered]

\definecolor{dkgreen}{rgb}{0,0.6,0}
\definecolor{gray}{rgb}{0.5,0.5,0.5}
\definecolor{mauve}{rgb}{0.58,0,0.82}


% Code
\usepackage{courier} %% Sets font for listing as Courier.
\usepackage{listings, xcolor}
\lstset{
	tabsize = 4, %% set tab space width
	showstringspaces = false, %% prevent space marking in strings, string is defined as the text that is generally printed directly to the console
	numbers = left, %% display line numbers on the left
	commentstyle = \color{red}, %% set comment color
	keywordstyle = \color{blue}, %% set keyword color
	stringstyle = \color{red}, %% set string color
	rulecolor = \color{black}, %% set frame color to avoid being affected by text color
	basicstyle = \small \ttfamily , %% set listing font and size
	breaklines = true, %% enable line breaking
	numberstyle = \tiny,
}

%Page Number
\addtobeamertemplate{navigation symbols}{}{%
	\usebeamerfont{footline}%
	\usebeamercolor[fg]{footline}%
	\hspace{1em}%
	\insertframenumber/\inserttotalframenumber
}
\expandafter\def\expandafter\insertshorttitle\expandafter{%
	\insertshorttitle\hfill%
	\insertframenumber\,/\,\inserttotalframenumber}
\begin{document}
	\author{Ms. Sonam Wangmo}
	\title{ITS202: Algorithms and Data Structure}
	\subtitle{Graphs}
	\institute{
		\textcolor{blue}{Gyalpozhing College of Information Technology \\ Royal University of Bhutan} \\
		\vspace{0.5cm}
	}
	%\date{}
	\setbeamercovered{transparent}
	%\setbeamertemplate{navigation symbols}{}
	\begin{frame}[plain]
		\maketitle
	\end{frame}

  \begin{frame}
 	\frametitle	{Graphs} 
 	\begin{block}{Graph}
 		A graph is a set of vertices and a collection of edges that each connect a pair of vertices.
 	\end{block} 
    
 \end{frame}

  \begin{frame}
	\frametitle	{Graphs} 
	\alert{Example of Graph: Facebook}
	\begin{figure}
		\centering
		\includegraphics[width=1.05\linewidth]{"Screenshot 2020-11-30 at 8.56.36 PM"}
		\label{fig:screenshot-2020-11-30-at-8}
	\end{figure}	
\end{frame}

  \begin{frame}
	\frametitle	{Graph Applications} 
	\begin{figure}
		\centering
		\includegraphics[width=0.9\linewidth]{"Screenshot 2020-11-30 at 9.00.32 PM"}
		\label{fig:screenshot-2020-11-30-at-9}
	\end{figure}		
\end{frame}

  \begin{frame}
	\frametitle	{Graph Terminology} 
	\textcolor{blue}{Path: } Sequence of vertices connected by edges. \\
	\textcolor{blue}{Cycle: } Path whose first and last vertices are the same.\\
	Two vertices are \alert{connected} if there is a path between them.	
		
\end{frame}

  \begin{frame}
	\frametitle	{Graph Terminology} 
	\begin{figure}
		\centering
		\includegraphics[width=0.6\linewidth]{"Screenshot 2020-11-30 at 9.04.08 PM"}
		\caption{}
		\label{fig:screenshot-2020-11-30-at-9}
	\end{figure}	
\end{frame}

  \begin{frame}
	\frametitle	{Some graph-processing problems} 
    \begin{figure}
    	\centering
    	\includegraphics[width=0.95\linewidth]{"Screenshot 2020-11-30 at 9.07.32 PM"}
    	\label{fig:screenshot-2020-11-30-at-9}
    \end{figure}  
\end{frame}

  \begin{frame}
	\frametitle	{Undirected Graph: Graph API} 
    \begin{figure}
    	\centering
    	\includegraphics[width=1.05\linewidth]{"Screenshot 2020-11-30 at 9.47.06 PM"}
    	\label{fig:screenshot-2020-11-30-at-9}
    \end{figure}
\end{frame}

 \begin{frame}
	\frametitle	{Undirected Graph: Graph API} 
    \begin{figure}
    	\centering
    	\includegraphics[width=1\linewidth]{"Screenshot 2020-11-30 at 9.47.22 PM"}
    	\label{fig:screenshot-2020-11-30-at-9}
    \end{figure}   
\end{frame}

\begin{frame}
	\frametitle	{Graph API: Sample Client} 
    \begin{figure}
    	\centering
    	\includegraphics[width=1.05\linewidth]{"Screenshot 2020-11-30 at 9.55.24 PM"}
    	\label{fig:screenshot-2020-11-30-at-9}
    \end{figure}   
\end{frame}

\begin{frame}
	\frametitle	{Graph API: Sample Client} 
	\begin{figure}
		\centering
		\includegraphics[width=1\linewidth]{"Screenshot 2020-11-30 at 9.59.34 PM"}
		\label{fig:screenshot-2020-11-30-at-9}
	\end{figure}	
\end{frame}

\begin{frame}
	\frametitle	{Graph Representation} 
	\begin{enumerate}
		\item Adjacency-matrix graph representation
		\item Adjacency-list graph representation
	\end{enumerate}
\end{frame}

\begin{frame}
	\frametitle	{Graph Representation: Adjacency-matrix graph representation} 
    Maintain a two-dimensional V-by-V boolean array;\\
    for each edge v–w in graph: adj[v][w] = adj[w][v] = true.
    \begin{figure}
    	\centering
    	\includegraphics[width=0.9\linewidth]{"Screenshot 2020-11-30 at 10.25.49 PM"}
    	\label{fig:screenshot-2020-11-30-at-10}
    \end{figure}
    
\end{frame}
\begin{frame}
	\frametitle	{Graph Representation: Adjacency-list graph representation} 
	Maintain vertex-indexed array of lists.
	\begin{figure}
		\centering
		\includegraphics[width=0.4\linewidth]{"Screenshot 2020-11-30 at 10.29.20 PM"}
		\label{fig:screenshot-2020-11-30-at-10}
	\end{figure}	
\end{frame}

\begin{frame}
	\frametitle	{Graph Traversal: Depth-first Search} 
     \alert{Maze Exploration}
     \begin{figure}
     	\centering
     	\includegraphics[width=0.8\linewidth]{"Screenshot 2020-12-01 at 6.10.23 AM"}
     	\label{fig:screenshot-2020-12-01-at-6}
     \end{figure}    
\end{frame}

\begin{frame}
	\frametitle	{Graph Traversal: Depth-first Search} 
    \begin{figure}
    	\centering
    	\includegraphics[width=0.7\linewidth]{"Screenshot 2020-12-01 at 6.12.06 AM"}
    	\label{fig:screenshot-2020-12-01-at-6}
    \end{figure}    
\end{frame}

\begin{frame}
	\frametitle	{Graph Traversal: Depth-first Search} 
	\alert{Goal}. Systematically traverse a graph.  
	\begin{figure}
		\centering
		\includegraphics[width=0.8\linewidth]{"Screenshot 2020-12-01 at 6.16.27 AM"}
		\label{fig:screenshot-2020-12-01-at-6}
	\end{figure}
	\alert{Typical applications}
	\begin{itemize}
		\item Find all vertices connected to a given source vertex. 
		\item Find a path between two vertices.
	\end{itemize}	
\end{frame}

\begin{frame}
	\frametitle	{Depth-first Search Demo} 
	\begin{figure}
		\centering
		\includegraphics[width=1\linewidth]{"Screenshot 2020-12-01 at 6.20.04 AM"}
		\label{fig:screenshot-2020-12-01-at-6}
	\end{figure}	
\end{frame}

\begin{frame}
	\frametitle	{Depth-first Search Demo} 
    \begin{figure}
    	\centering
    	\includegraphics[width=1\linewidth]{"Screenshot 2020-12-01 at 6.21.31 AM"}
    	\label{fig:screenshot-2020-12-01-at-6}
    \end{figure}    	
\end{frame}

\begin{frame}
	\frametitle	{Depth-first Search: Data Structure} 
	\alert{Data structures.}
     \begin{itemize}
     	\item Boolean array marked[] to mark visited vertices.
     	\item Integer array edgeTo[] to keep track of paths. (edgeTo[w] == v) means that edge v-w taken to visit w for first time .	
     \end{itemize}
\end{frame}

\begin{frame}
	\frametitle	{Depth-first search application: preparing for a date} 
    \begin{figure}
    	\centering
    	\includegraphics[width=0.9\linewidth]{"Screenshot 2020-12-01 at 6.26.40 AM"}
    	\label{fig:screenshot-2020-12-01-at-6}
    \end{figure}  
\end{frame}

\begin{frame}
	\frametitle	{Depth-first search application: preparing for a date} 
    \alert{Maze Application that we discussed}
\end{frame}

\begin{frame}
	\frametitle	{Graph Traversal: Breadth-first Search} 
    \begin{figure}
    	\centering
    	\includegraphics[width=0.9\linewidth]{"Screenshot 2020-12-01 at 11.08.51 AM"}
    	\label{fig:screenshot-2020-12-01-at-11}
    \end{figure}   
\end{frame}

\begin{frame}
	\frametitle	{Graph Traversal: Breadth-first Search} 
    \begin{figure}
    	\centering
    	\includegraphics[width=0.95\linewidth]{"Screenshot 2020-12-01 at 11.11.15 AM"}
    	\label{fig:screenshot-2020-12-01-at-11}
    \end{figure}   
\end{frame}

\begin{frame}
	\frametitle	{Graph Traversal: Breadth-first Search} 
    \alert{Algorithm}
    \begin{figure}
    	\centering
    	\includegraphics[width=0.9\linewidth]{"Screenshot 2020-12-01 at 11.13.02 AM"}
    	\label{fig:screenshot-2020-12-01-at-11}
    \end{figure}    
\end{frame}

\begin{frame}
	\frametitle	{Graph Traversal: Breadth-first Search} 
    \begin{figure}
    	\centering
    	\includegraphics[width=1\linewidth]{"Screenshot 2020-12-01 at 11.14.39 AM"}
    	\label{fig:screenshot-2020-12-01-at-11}
    \end{figure}      
\end{frame}

\begin{frame}
	\frametitle	{Breadth-first search application: routing} 
	  \begin{figure}
	  	\centering
	  	\includegraphics[width=0.9\linewidth]{"Screenshot 2020-12-01 at 11.22.42 AM"}
	  	\label{fig:screenshot-2020-12-01-at-11}
	  \end{figure}	  
\end{frame}

\begin{frame}
	\frametitle	{Connected Componets} 
    \alert{Def.} Vertices v and w are \alert{connected} if there is a path between them.  \\
    \alert{Def.} A \alert{connected component} is a maximal set of connected vertices.
    \begin{figure}
    	\centering
    	\includegraphics[width=0.7\linewidth]{"Screenshot 2020-12-01 at 12.15.00 PM"}
    	\label{fig:screenshot-2020-12-01-at-12}
    \end{figure}
\end{frame}

\begin{frame}
	\frametitle	{Graph Traversal Summary} 
	\begin{figure}
		\centering
		\includegraphics[width=1\linewidth]{"Screenshot 2020-12-01 at 12.20.47 PM"}
		\label{fig:screenshot-2020-12-01-at-12}
	\end{figure}	
\end{frame}

\end{document}