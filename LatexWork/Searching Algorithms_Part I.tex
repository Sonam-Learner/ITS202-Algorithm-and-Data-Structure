\documentclass[11pt]{beamer}
\usepackage[utf8]{inputenc}
\usepackage[T1]{fontenc}
\usepackage{lmodern}
\usetheme{Copenhagen}
\usepackage{listings}
\usepackage{color}
\usepackage{caption}
\usepackage{graphicx}
\usepackage{outlines}
\setbeamertemplate{caption}[numbered]

\definecolor{dkgreen}{rgb}{0,0.6,0}
\definecolor{gray}{rgb}{0.5,0.5,0.5}
\definecolor{mauve}{rgb}{0.58,0,0.82}


% Code
\usepackage{courier} %% Sets font for listing as Courier.
\usepackage{listings, xcolor}
\lstset{
	tabsize = 4, %% set tab space width
	showstringspaces = false, %% prevent space marking in strings, string is defined as the text that is generally printed directly to the console
	numbers = left, %% display line numbers on the left
	commentstyle = \color{red}, %% set comment color
	keywordstyle = \color{blue}, %% set keyword color
	stringstyle = \color{red}, %% set string color
	rulecolor = \color{black}, %% set frame color to avoid being affected by text color
	basicstyle = \small \ttfamily , %% set listing font and size
	breaklines = true, %% enable line breaking
	numberstyle = \tiny,
}


\begin{document}
	\author{Ms. Sonam Wangmo}
	\title{ITS202: Algorithms and Data Structures}
	\subtitle{Searching Algorithms}
	\institute{
		\textcolor{blue}{Gyalpozhing College of Information Technology \\ Royal University of Bhutan} \\
		\vspace{0.5cm}
	}

	%\date{}
	\setbeamercovered{transparent}
	%\setbeamertemplate{navigation symbols}{}
	\begin{frame}[plain]
		\maketitle
	\end{frame}
	\begin{frame}
		\frametitle{Searching Algorithms}
		Searching Algorithms that we will study are: 
		\begin{enumerate}
			\item Linear Search
			\item Binary Search
		\end{enumerate}		 
	\end{frame}
	
	\begin{frame}
		\frametitle{Linear Search}
			Idea of the algorithm is to iterate across the array from left to right, searching for a specified element. \\
			\textbf{In Psuedocode}
			\begin{enumerate}
				\item Repeat, starting at the first element:
				\begin{itemize}
					\item If the first element is what you’re looking for (the target), stop.
					\item Otherwise, move to the next element.
				\end{itemize} 
				\item For i from 0 to n-1
				If i’th element is the element to be searched
				\begin{itemize}
						\item Return true \\
				\end{itemize}
				Return false
				\end{enumerate}		 
	\end{frame}

\begin{frame}
	\frametitle{Binary Search}
	 \begin{block}{Concepts}
	 The idea of algorithm is to divide and conquer, reducing the search by half each time while trying to find the target element.
	\textbf{Important:}  To perform binary search in an array, the array needs to be sorted always.
	 \end{block}
\end{frame}

\begin{frame}
	\frametitle{Binary Search}
	\begin{block}{Psuedocode}
	\begin{enumerate}
		\item Repeat until the array is of size 0
		\begin{itemize}
			\item Calculate the middle point of the current array.
			\item If the target is at the middle, stop.
			\item Otherwise, if the target is less than what’s at the middle, repeat, changing the end
			point to just to the left of the middle.
			\item Otherwise, if the target is greater than what’s at the middle, repeat, changing the start point to just to the right of the middle.
		\end{itemize}
	\end{enumerate}	
\end{block}
\end{frame}
\end{document}