\documentclass[11pt]{beamer}
\usepackage[utf8]{inputenc}
\usepackage[T1]{fontenc}
\usepackage{lmodern}
\usetheme{Copenhagen}
\usepackage{listings}
\usepackage{color}
\usepackage{caption}
\usepackage{graphicx}
\usepackage{outlines}
\setbeamertemplate{caption}[numbered]

\definecolor{dkgreen}{rgb}{0,0.6,0}
\definecolor{gray}{rgb}{0.5,0.5,0.5}
\definecolor{mauve}{rgb}{0.58,0,0.82}


% Code
\usepackage{courier} %% Sets font for listing as Courier.
\usepackage{listings, xcolor}
\lstset{
	tabsize = 4, %% set tab space width
	showstringspaces = false, %% prevent space marking in strings, string is defined as the text that is generally printed directly to the console
	numbers = left, %% display line numbers on the left
	commentstyle = \color{red}, %% set comment color
	keywordstyle = \color{blue}, %% set keyword color
	stringstyle = \color{red}, %% set string color
	rulecolor = \color{black}, %% set frame color to avoid being affected by text color
	basicstyle = \small \ttfamily , %% set listing font and size
	breaklines = true, %% enable line breaking
	numberstyle = \tiny,
}


\begin{document}
	\author{Ms. Sonam Wangmo}
	\title{ITS202: Algorithms and Data Structures}
	\subtitle{Advanced Data Structures}
	\institute{
		\textcolor{blue}{Gyalpozhing College of Information Technology \\ Royal University of Bhutan} \\
		\vspace{0.5cm}
	}
	%\date{}
	\setbeamercovered{transparent}
	%\setbeamertemplate{navigation symbols}{}
	\begin{frame}[plain]
		\maketitle
	\end{frame}
	\begin{frame}
		\frametitle{Trees}
		\begin{block}{Formal Defination}
			we define a tree T as a set of nodes storing elements such that the nodes have a parent-child relationship that satisfies the following properties:
			\begin{itemize}
				\item If T is nonempty, it has a special node, called the root of T , that has no parent. 
				\item Each node v of T different from the root has a unique parent node w; every node with parent w is a child of w.
			\end{itemize}
		\end{block}	
	\end{frame}

		\begin{frame}
		\frametitle{Trees}
		\begin{block}{Terms Used in Trees}
			Two nodes that are children of the same parent are \textbf{siblings}. A node v is \textbf{external} if v has no children. A node v is \textbf{internal} if it has one or more children. External nodes are also known as \textbf{leaves}.\\	
		\end{block}
	\end{frame}

	\begin{frame}	
		\frametitle{Trees}
		\begin{figure}
			\centering
			\includegraphics[width=0.8\linewidth]{"../../../../../../Desktop/Screenshot 2020-10-27 at 8.25.40 PM"}
			\caption{Tree representing a portion of a file system.}
			\label{fig:screenshot-2020-10-27-at-8}
		\end{figure}	
	\end{frame}

	\begin{frame}
	\frametitle{Trees}
		A node u is an ancestor of a node v if u is an \textbf{ancestor} of the parent of v. \\
		Conversely, we say that a node v is a \textbf{descendant} of a node u if u is an ancestor of v. \\
		\textbf{For example}, in Figure 1, cs252/ is an ancestor of papers/, and pr3 is a descendant of cs016/. \\
		The \textbf{subtree} of T rooted at a node v is the tree consisting of all the descendants of v in T (including v itself). In Figure 1, the subtree rooted at cs016/ consists of the nodes cs016/, grades, homeworks/, programs/, hw1, hw2, hw3, pr1, pr2, and pr3.
    \end{frame}

    \begin{frame}
    	\frametitle{Trees}
    		\begin{block}{	Edges and Paths in Trees}
    			An \textbf{edge} of tree T is a pair of nodes (u,v) such that u is the parent of v, or vice versa. A \textbf{path} of T is a sequence of nodes such that any two consecutive nodes in the sequence form an edge. //
    			\textbf{For example}, the tree in Figure 1 contains the path (cs252/, projects/, demos/, market).
    	\end{block}
    \end{frame}

     \begin{frame}
   	\frametitle{The Tree Abstract Data Type}
    \begin{itemize}
    	\item \textcolor{blue}{getElement():}Returns the element stored at this position. 
    	\item \textcolor{blue}{root():}Returns the position of the root of the tree (or null if empty).
    	\item \textcolor{blue}{parent(p):}Returns the position of the parent of position p (or null if p is the root).
    	\item \textcolor{blue}{children(p):}Returns an iterable collection containing the children of position p (if any).
    	\item \textcolor{blue}{numChildren(p):}Returns the number of children of position p.
    	\item \textcolor{blue}{isInternal(p):}Returns true if position p has at least one child.
    \end{itemize}
   \end{frame}
  
      \begin{frame}
    	\frametitle{The Tree Abstract Data Type}
    	\begin{itemize}
    	\item \textcolor{blue}{isExternal(p):}Returns true if position p does not have any children.
    	\item \textcolor{blue}{isRoot(p):}Returns true if position p is the root of the tree.
    	\item \textcolor{blue}{size():}Returns the number of positions (and hence elements) that are contained in the tree.
    	\item \textcolor{blue}{isEmpty():}Returns true if the tree does not contain any positions (and thus no elements).
    	\item \textcolor{blue}{iterator():}Returns an iterator for all elements in the tree (so that the tree itself is Iterable).
    	\item \textcolor{blue}{positions():}Returns an iterable collection of all positions of the tree.
    	\end{itemize}
    \end{frame}

   \begin{frame}
  	\frametitle{Computing Depth and Height}
  	Let p be a position within tree T. The depth of p is the number of ancestors of p, other than p itself. \\
  	 Note that this definition implies that the depth of the root of T is 0. The depth of p can also be recursively defined as follows:
  	 \begin{itemize}
  	 	\item If p is the root,then the depth of p is 0.
  	 	\item Otherwise, the depth of p is one plus the depth of the parent of p.
  	 \end{itemize}
  \end{frame}

	\begin{frame}
		\frametitle{Computing Depth and Height}
	Formally, we define the height of a position p in a tree T as follows:
		\begin{itemize}
			\item  If p is a leaf,then the height of p is 0.
			\item Otherwise, the height of p is one more than the maximum of the heights of p’s children.
		\end{itemize}
	\end{frame}

   	\begin{frame}
   	\frametitle{Binary Trees}
   A binary tree is an ordered tree with the following properties:
   	\begin{itemize}
   		\item  Every node has at most two children.
   		\item Each child node is labeled as being either a left child or a right child.
   		\item A left child precedes a right child in the order of children of a node.\\
   	\end{itemize}
   \begin{alertblock}{Note}
   	The subtree rooted at a left or right child of an internal node v is called a left subtree or right subtree, respectively, of v
   \end{alertblock}
   \end{frame}

    \begin{frame}
  	\frametitle{The Binary Tree Abstract Data Type}
  	As an abstract data type, a binary tree is a specialization of a tree that supports three additional accessor methods:
  	\begin{itemize}
  		\item \textcolor{blue}{left(p):}Returns the position of the left child of p (or null if p has no left child).
  		\item \textcolor{blue}{right(p):}Returns the position of the right child of p (or null if p has no right child).
  		\item \textcolor{blue}{sibling(p):}Returns the position of the sibling of p
  		(or null if p has no sibling).
  	\end{itemize}
  \end{frame}

	 \begin{frame}
	 	\frametitle{Properties of Binary Trees}
 	    \begin{figure}
 	    	\centering
 	    	\includegraphics[width=0.87\linewidth]{"../../../../../../Desktop/Screenshot 2020-10-27 at 11.24.05 PM"}
 	    	\caption{Maximum number of nodes in the levels of a binary tree.}
 	    	\label{fig:screenshot-2020-10-27-at-11}
 	    \end{figure}   
	 \end{frame}
     
     \begin{frame}
     	\frametitle{Properties of Binary Trees}
     	\begin{alertblock}{Note}
     		We denote the set of all nodes of a tree T at the same depth d as \textbf{level} d of T . \\
     		In a binary tree, level 0 has at most one node (the root), level 1 has at most two nodes (the children of the root), level 2 has at most four nodes, and so on.  In general, level d has at most $2^{d} $ nodes.
     	\end{alertblock}   	
     \end{frame}
 
 	 \begin{frame}
 		\frametitle{Tree Traversal Algorithms}
 	     A traversal of a tree T is a systematic way of accessing, or “visiting,” all the positions of T. \\
 	     \begin{exampleblock}{{Three way tree traversal of a Binary Tree}}
 	     	\begin{enumerate}
 	     		\item Inorder Traversal steps
 	     		\item Preorder Traversal steps
 	     		\item Postorder Traversal steps
 	     	\end{enumerate}
 	     \end{exampleblock}   
         \begin{figure}
         	\centering
         	\includegraphics[width=0.3\linewidth]{"../../../../../../Desktop/Screenshot 2020-10-28 at 9.24.11 AM"}
         	\caption{Typical Tree}
         	\label{fig:screenshot-2020-10-28-at-9}
         \end{figure}       	
 	\end{frame}
 
    \begin{frame}
    	\frametitle{Tree Traversal Algorithms}
    	\begin{figure}
    		\centering
    		\includegraphics[width=0.65\linewidth]{"../../../../../../Desktop/Screenshot 2020-10-28 at 9.21.20 AM"}
    		\caption{Tree Traversals}
    		\label{fig:screenshot-2020-10-28-at-9}
    	\end{figure}
    \end{frame}

    \begin{frame}
    	\frametitle{Tree Traversal Algorithms}  	
    	\begin{block}{InOrder Traversal Steps}
    		\begin{enumerate}
    			\item Visit the left sub-tree in
    			inorder
    			\item Visit the root
    			\item Visit the right sub-tree inorder
    		\end{enumerate}
    	\end{block}   
    	\begin{block}{PreOrder Traversal Steps}
    		\begin{enumerate}
    			\item Visit the root
    			\item Visit the left sub-tree in
    			preorder
    			\item Visit the right sub-tree preorder
    		\end{enumerate}
    	\end{block}  
    	\begin{block}{PostOrder Traversal Steps}
    		\begin{enumerate}
    			\item Visit the left sub-tree in
    			postorder
    			\item Visit the right sub-tree postorder
    			\item Visit the root
    		\end{enumerate}
    	\end{block}  	
    \end{frame}
    
    \begin{frame}
    	\frametitle{Tree Traversals}  	
    	 \begin{figure}
    	 	\centering
    	 	\includegraphics[width=0.99\linewidth]{"../../../../../../Desktop/Screenshot 2020-10-28 at 9.33.34 AM"}
    	 	\caption{Example 1}
    	 	\label{fig:screenshot-2020-10-28-at-9}
    	 \end{figure} 	 
    \end{frame}

      \begin{frame}
    	\frametitle{Tree Traversals}  	
    	\begin{figure}
    		\centering
    		\includegraphics[width=0.99\linewidth]{"../../../../../../Desktop/Screenshot 2020-10-28 at 9.36.36 AM"}
    		\caption{Example 2}
    		\label{fig:screenshot-2020-10-28-at-9}
    	\end{figure}	 
    \end{frame}
    
     \begin{frame}
    	\frametitle{Tree Traversals}  	
    	\begin{figure}
    		\centering
    		\includegraphics[width=0.99\linewidth]{"../../../../../../Desktop/Screenshot 2020-10-28 at 9.39.07 AM"}
    		\caption{Solution of Example 2}
    		\label{fig:screenshot-2020-10-28-at-9}
    	\end{figure}
    	
    \end{frame}

\end{document}