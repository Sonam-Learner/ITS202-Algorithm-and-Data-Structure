\documentclass[11pt]{beamer}
\usepackage[utf8]{inputenc}
\usepackage[T1]{fontenc}
\usepackage{lmodern}
\usetheme{Copenhagen}
\usepackage{listings}
\usepackage{color}
\usepackage{caption}
\usepackage{graphicx}
\usepackage{outlines}
\setbeamertemplate{caption}[numbered]

\definecolor{dkgreen}{rgb}{0,0.6,0}
\definecolor{gray}{rgb}{0.5,0.5,0.5}
\definecolor{mauve}{rgb}{0.58,0,0.82}


% Code
\usepackage{courier} %% Sets font for listing as Courier.
\usepackage{listings, xcolor}
\lstset{
	tabsize = 4, %% set tab space width
	showstringspaces = false, %% prevent space marking in strings, string is defined as the text that is generally printed directly to the console
	numbers = left, %% display line numbers on the left
	commentstyle = \color{red}, %% set comment color
	keywordstyle = \color{blue}, %% set keyword color
	stringstyle = \color{red}, %% set string color
	rulecolor = \color{black}, %% set frame color to avoid being affected by text color
	basicstyle = \small \ttfamily , %% set listing font and size
	breaklines = true, %% enable line breaking
	numberstyle = \tiny,
}


\begin{document}
	\author{Ms. Sonam Wangmo}
	\title{ITS202: Algorithms and Data Structures}
	\subtitle{Symbol Tables}
	\institute{
		\textcolor{blue}{Gyalpozhing College of Information Technology \\ Royal University of Bhutan} \\
		\vspace{0.5cm}
	}
	%\date{}
	\setbeamercovered{transparent}
	%\setbeamertemplate{navigation symbols}{}
	\begin{frame}[plain]
		\maketitle
	\end{frame}
	\begin{frame}
		\frametitle{Symbol Tables}
			A \textbf{symbol table} is a data structure for key-value pairs that supports two operations: insert (put) a new pair into the table and search for (get) the value associated with a given key.\\ 
	\end{frame}

\begin{frame}
	\frametitle{Symbol Tables}
	\begin{figure}
		\centering
		\includegraphics[width=1\linewidth]{"../../../../../../Desktop/Screenshot 2020-10-21 at 9.14.44 AM"}
		\caption{Example}
		\label{fig:screenshot-2020-10-21-at-9}
	\end{figure}	
\end{frame}

\begin{frame}
	\frametitle{Symbol Tables}
	 \begin{figure}
	 	\centering
	 	\includegraphics[width=1\linewidth]{"../../../../../../Desktop/Screenshot 2020-10-20 at 10.36.51 PM"}
	 	\caption{Typical symbol-table applications}
	 	\label{fig:screenshot-2020-10-20-at-10}
	 \end{figure}
	 
\end{frame}

\begin{frame}
	\frametitle{Symbol Tables}
	\begin{figure}
		\centering
		\includegraphics[width=1\linewidth]{"../../../../../../Desktop/Screenshot 2020-10-20 at 10.56.59 PM"}
		\caption{API for a generic basic symbol table}
		\label{fig:screenshot-2020-10-20-at-10}
	\end{figure}	
\end{frame}

\begin{frame}
	\frametitle{Elementary Implementation}
	\begin{itemize}
		\item Sequential search (unordered list)
		\item Binary search (ordered array)
	\end{itemize}
\end{frame}

\begin{frame}
	\frametitle{Elementary Implementation: Sequential search in LinkedList }
	\begin{figure}
		\centering
		\includegraphics[width=0.9\linewidth]{"../../../../../../Desktop/Screenshot 2020-10-21 at 1.10.21 PM"}
		\caption{Trace of linked-list ST implementation}
		\label{fig:screenshot-2020-10-21-at-1}
	\end{figure}	
\end{frame}

\begin{frame}
	\frametitle{Elementary Implementation: Sequential search in LinkedList }
    \begin{figure}
    	\centering
    	\includegraphics[width=0.7\linewidth]{"../../../../../../Desktop/Screenshot 2020-10-21 at 1.12.16 PM"}
    	\caption{Summary}
    	\label{fig:screenshot-2020-10-21-at-1}
    \end{figure}
\end{frame}

\begin{frame}
	\frametitle{Elementary Implementation: Binary search in an Ordered array}
   \begin{figure}
   	\centering
   	\includegraphics[width=0.8\linewidth]{"../../../../../../Desktop/Screenshot 2020-10-21 at 8.26.54 PM"}
   	\caption{Search}
   	\label{fig:screenshot-2020-10-21-at-8}
   \end{figure} 
\end{frame}

\begin{frame}
	\frametitle{Elementary Implementation: Binary search in an Ordered array}
   \begin{figure}
   	\centering
   	\includegraphics[width=0.9\linewidth]{"../../../../../../Desktop/Screenshot 2020-10-21 at 8.32.47 PM"}
   	\caption{Insertion}
   	\label{fig:screenshot-2020-10-21-at-8}
   \end{figure} 
\end{frame}

\begin{frame}
	\frametitle{Elementary Implementation: Binary search in an Ordered array}
	\begin{figure}
		\centering
		\includegraphics[width=0.7\linewidth]{"../../../../../../Desktop/Screenshot 2020-10-21 at 9.54.16 PM"}
		\caption{Summary}
		\label{fig:screenshot-2020-10-21-at-9}
	\end{figure}	
\end{frame}

\begin{frame}
	\frametitle{Ordered Operations}
    \begin{figure}
    	\centering
    	\includegraphics[width=0.75\linewidth]{"../../../../../../Desktop/Screenshot 2020-10-21 at 9.57.23 PM"}
    	\caption{Example}
    	\label{fig:screenshot-2020-10-21-at-9}
    \end{figure}    	
\end{frame}

\begin{frame}
	\frametitle{Ordered Operations}
	\begin{figure}
		\centering
		\includegraphics[width=0.8\linewidth]{"../../../../../../Desktop/Screenshot 2020-10-21 at 9.59.05 PM"}
		\caption{}
		\label{fig:screenshot-2020-10-21-at-9}
	\end{figure}	
\end{frame}



\begin{frame}
	\frametitle{Binary search: ordered symbol table operations summary}
	\begin{figure}
		\centering
		\includegraphics[width=0.6\linewidth]{"../../../../../../Desktop/Screenshot 2020-10-21 at 10.00.27 PM"}
		\caption{order of growth of the running time for ordered symbol table operations}
		\label{fig:screenshot-2020-10-21-at-10}
	\end{figure}	
\end{frame}



\end{document}