\documentclass[11pt]{beamer}
\usepackage[utf8]{inputenc}
\usepackage[T1]{fontenc}
\usepackage{lmodern}
\usetheme{Copenhagen}
\usepackage{listings}
\usepackage{color}
\usepackage{caption}
\usepackage{graphicx}
\usepackage{outlines}
\setbeamertemplate{caption}[numbered]

\definecolor{dkgreen}{rgb}{0,0.6,0}
\definecolor{gray}{rgb}{0.5,0.5,0.5}
\definecolor{mauve}{rgb}{0.58,0,0.82}


% Code
\usepackage{courier} %% Sets font for listing as Courier.
\usepackage{listings, xcolor}
\lstset{
	tabsize = 4, %% set tab space width
	showstringspaces = false, %% prevent space marking in strings, string is defined as the text that is generally printed directly to the console
	numbers = left, %% display line numbers on the left
	commentstyle = \color{red}, %% set comment color
	keywordstyle = \color{blue}, %% set keyword color
	stringstyle = \color{red}, %% set string color
	rulecolor = \color{black}, %% set frame color to avoid being affected by text color
	basicstyle = \small \ttfamily , %% set listing font and size
	breaklines = true, %% enable line breaking
	numberstyle = \tiny,
}


\begin{document}
	\author{Ms. Sonam Wangmo}
	\title{ITS202: Algorithms and Data Structures}
	\subtitle{String Sorts}
	\institute{
		\textcolor{blue}{Gyalpozhing College of Information Technology \\ Royal University of Bhutan} \\
		\vspace{0.5cm}
	}

	%\date{}
	\setbeamercovered{transparent}
	%\setbeamertemplate{navigation symbols}{}
	\begin{frame}[plain]
		\maketitle
	\end{frame}
	
	\begin{frame}
		\frametitle{Radix Sort}
			We consider two fundamentally different approaches to string sorting.
			\begin{enumerate}
				\item LSD referred to as least-significant-digit (LSD) string sorts.
				\item MSD referred to as most-significant-digit (MSD) string sorts.
				\end{enumerate}		 
	\end{frame}

\begin{frame}
	\frametitle{LSD radix sort}
	 \begin{block}{Concepts}
	 \begin{itemize}
	 	\item Consider characters from right to left.
	 	\item Sort using dth character as the key (Using key-indexed counting)
	 	\item LSD sorts fixed length strings in ascending order.
	 	\item Sorting such string can be done using key-indexed counting. 
	 	\item If the strings are each of length W,
	 	we sort the strings W times with key-indexed counting, using
	 	each of the positions as the key, proceeding from right to left.
	 \end{itemize}
	 \end{block}
\end{frame}

%Radix
\begin{frame}[fragile]
    \frametitle{LSD radix sort}
  
	\begin{lstlisting}[language=Java]		
	public class LSD
		{
			public static void sort(String[] a, int W)
			{
				//Sort a[] on leading W characters.
				int N = a.length; 
				int R = 256; 
				String[] aux = new String[N];
				for (int d = W-1; d >= 0; d--)
				{ 
				// Sort by key-indexed counting on dth char.
					int[] count = new int[R+1];   
			    // Compute frequency counts.
					for (int i = 0; i < N; i++)
					
				
	\end{lstlisting}
\end{frame}

%ContinueRadix
\begin{frame}[fragile]
	\frametitle{LSD radix sort}
	
	\begin{lstlisting}[language=Java]		
			    	count[a[i].charAt(d) + 1]++;
					for (int r = 0; r < R; r++)    
					// Transform counts to indices.
					count[r+1] += count[r];
					for (int i = 0; i < N; i++)    
					// Distribute.
					aux[count[a[i].charAt(d)]++] = a[i];
					for (int i = 0; i < N; i++)     
					// Copy back.
					a[i] = aux[i];
				} 
			}
		}
	\end{lstlisting}
\end{frame}
\begin{frame}[fragile]
	\frametitle{ LSD radix sort}
	\begin{flushleft}
		To sort an array a[] of strings that each have exactly W characters, we do W key-indexed counting sorts: one for each character position, proceeding from right to left.
	\end{flushleft}   
\end{frame}

\begin{frame}[fragile]
	\frametitle{LSD radix sort }
	\begin{figure}
		\centering
		\includegraphics[width=1.05\linewidth]{"../../../../../../Desktop/Screenshot 2020-10-18 at 10.37.36 PM"}
		\caption{LSD}
		\label{fig:screenshot-2020-10-18-at-10}
	\end{figure}	
\end{frame}

\begin{frame}
	\frametitle{LSD radix sort}
	\pause
	Q. What if strings are not all of same length?
\end{frame}

\begin{frame}
	\frametitle{MSD radix sort}
	\begin{block}{Concepts}
		\begin{itemize}
			\item Consider characters from left to right order.
			\item Sort using dth character as the key (Using key-indexed counting)
			\item A general-purpose string sort, where strings are not necessarily all the same length
			\item Sorting such string can be done using key-indexed counting. 
			\item If the strings are each of length W,
			we sort the strings W times with key-indexed counting, using
			each of the positions as the key, proceeding from right to left.
		\end{itemize}
	\end{block}
\end{frame}
\begin{frame}
	\frametitle{MSD radix sort}
	\begin{figure}
		\centering
		\includegraphics[width=0.85\linewidth]{"Screenshot 2020-10-20 at 9.52.19 AM"}
		\caption{MSD Sorting}
		\label{fig:screenshot-2020-10-20-at-9}
	\end{figure}
	
\end{frame}
\end{document}